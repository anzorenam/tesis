% !TEX root = tesis.tex

\chapter{Conclusión}
\chaptermark{Conclusión}

La operación del SciCRT será fundamental para comprender los mecanismos de aceleración en el Sol. El nuevo detector, además de su volumen activo mayor, tiene una mayores capacidades para identificar partículas y mejor resolución en energía; lo cual lo hace un excelente instrumento para estimar el espectro de energía de los neutrones solares. Desde \num{2015} el telescopio opera de manera estable, gracias a la instalación de electrónica de alta velocidad. Aunado a esto, hemos comenzado el diseño de unidades de electrónica FE, con vista a alcanzar el potencial del SciCRT como telescopio de neutrones solares.

Para lograr este objetivo hemos diseñado un técnica de simulación que incluye todos los procesos importantes en la generación de las señales, alimentando a la entrada de la simulación muones de rayos cósmicos secundarios, con el fin de analizar su deposición de energía e información temporal. A partir de esto hemos logrado diseñar y construir prototipos de la nueva electrónica, con la ventaja de tener una solución efectiva en costo y potencia. La validación de la metodología la realizamos construyendo una versión de \num{16} canales de preamplificador/formador, incluyendo la descripción de las funciones de discriminación, TDC y transferencia de datos en un FPGA. La prueba de este sistema la realizamos mediante un pulsador LED.

Una prueba extensiva se realizo en un experimento de alta montaña, con el propósito de extraer las características de la señal real y compararlas con la simulación. Todos los resultados obtenidos se encuentran en buena concordancia con la simulación. Además, la aplicación de este análisis nos permitió determinar no-linealidades en la respuesta del detector, y encontrar una relación importante entre la señal de TOT y la distancia con respecto al sensor óptico. Por otro lado, el estudio de la longitud de atenuación en las fibras usando datos experimentales nos permitió la calibración de la simulación MC para incluir propiedades ópticas de las barras de centelleo y fibras WLS que habían sido analizadas con anterioridad. Con respecto a las características temporales de la señales del detector, a través de nuestro análisis hemos determinado la necesidad de estudiar a profundidad la dependencia de varios parámetros de la electrónica con respecto a la temperatura, lo cual puede afectar la resolución del telescopio.

Una observación final con respecto a la desarrollo de la electrónica es que, bajo las restricciones actuales de nuestro análisis, la distribución temporal de las señales producidas por neutrones en las barras de centelleo no se ha estudiado; y será motivo de futuras investigaciones debido a su importancia.

Posteriormente, en la parte final de esta tesis presentamos un estudio del desempeño del SciCRT como telescopio de rayos cósmicos y un análisis de su operación estable. El estudio de desempeño lo realizamos comprando las tasas de eventos del experimento y una simulación MC que incluye siete especies de partículas. Además, la simulación también considera las distribuciones de energía y angulares de las componentes de la radiación cósmica, así como su abundancia en el flujo total. La tasa de eventos obtenida a través de la simulación es de: \SI{3132.31(9480)}{eventos \per\minute}. Por otra parte, al analizar la tasa de eventos del  experimento encontramos una contaminación por eventos de menor energía a la del umbral. Posterior al filtrado de los eventos accidentales, obtuvimos una tasa de eventos de \SI{3178.40(177)}{eventos \per\minute}, con lo cual concluimos que ambas concuerdan dentro de las incertidumbres asociadas.

En el estudio de estabilidad nos concentramos en analizar las variaciones de la ganancia de los \num{28} MAPMT que componen el SB3 del SciCRT. Con este objetivo nos enfocamos en analizar tres meses de datos, de Diciembre de \num{2019} a Febrero de \num{2020}, lo cual presentó un reto técnico debido al gran volumen de datos que hay que procesar. Como resultado de este estudio pudimos concluir que la ganancia de los MAPMT se ve afectada, principalmente por condiciones ambientales, pero debido a la instalación de infraestructura adecuada en el sitio la máxima variación observada es de solo \SI{2.5}{\percent}. Este resultado nos da garantía de que el SciCRT opera de manera estable, no obstante resalta la importancia de monitorizar sistemáticamente la ganancia del telescopio para conocer la incertidumbre asociada con la estimación de energía.

Ambos análisis (estabilidad y desempeño) apuntan a que el detector en su condición actual puede ser usado de manera confiable para observar partículas energéticas solares. Luego entonces es interés para la comunidad científica el análisis de los datos obtenidos por el SciCRT durante estos primeros años de operación.

Teniendo esta motivación realizamos un estudio de los datos obtenidos durante dos ráfagas solares de intensidad media (tipo M) y encontramos que hay buenas razones para concluir que el SciCRT observó partículas provenientes del Sol: neutrones y rayos $\gamma$. Es importante remarcar aquí, que aunque el método propuesto nos ha permitido hacer una primera aproximación al problema, el método tiene varias limitaciones y en ese sentido las técnicas empleadas en \cite{garcia20} presentan una mejor alternativa. Un futuro análisis de estos eventos, en donde se estudie con detalle el espectro de energía de las partículas detectadas, así como su dirección de arribo; no solo permitirá comprobar la detección sino también entender el escenario bajo el cual se aceleraron estas partículas. Esto notable dado que, ya que no solo el SciCRT no ha permite estudiar con mayor profundidad estos eventos, sino que además nos abre una nueva ventana al ser más sensible a neutrones de baja energía, en comparación con la generación previa de TNS.
