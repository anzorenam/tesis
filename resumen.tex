% !TEX root = tesis.tex

\pagestyle{empty}
\chapter{Resumen}

Los neutrones solares son un fenómeno único. Éstos son producidos en intensas explosiones solares, mediante la interacción del plasma en la atmósfera del Sol con iones acelerados. Luego entonces, una medición del espectro de energía de los neutrones confiere información del mecanismo de aceleración, ya que  los neutrones viajan por el medio interplanetario sin ser afectados por el campo magnético. Al propagarse por la atmósfera terrestres, éstos son fuertemente atenuados y solo una pequeña cantidad de ellos puede penetrar a cierta profundidad atmosférica. Ya que los neutrones tienen masa, el tiempo de propagación del Sol a la Tierra depende de su energía; lo cual hace de vital importancia medir el tiempo de arribo para poder descartar entre una emisión impulsiva o continua.

Con el objetivo de estudiar este fenómeno se desarrollo una red mundial de telescopios de neutrones, instalados en siete montañas, cercanas al ecuador. Esta red de telescopios observa al Sol \SI{24}{\hour} al día y ha sido capaz de detectar eventos relacionados con ráfagas solares. Sin embargo, el desarrollo de un nuevo tipo de detector con un volumen activo mayor es esencial para mejorar la significancia estadística y resolución en energía. Este nuevo tipo de detector es el SciCRT.

En \num{2013} instalamos el SciCRT en la cima del volcán Sierra Negra, Puebla, ($\ang{19.0}\mathbf{N}$, $\ang{97.3}\mathbf{W}$) para operar como un nuevo y mejorado telescopio de neutrones solares (y de muones). Comparado con la generación previa de detectores, el SciCRT tiene una mayor resolución en energía y es sensible a neutrones de menor energía. Además, dado que el telescopio permite registrar la trayectoria de las partículas y la energía depositada en el trayecto; nuevas técnicas de identificación de partículas se pueden implementar, mejorando la sensibilidad a los neutrones solares.

Por otra parte, si consideramos que solo están disponibles $3/8$ de la electrónica necesaria para la instalación completa del telescopio, el desarrollo de un sistema de adquisición de datos de alta velocidad se vuelve una prioridad para el experimento. Con esta motivación comencé el diseño de nueva electrónica rápida para el telescopio. El desarrollo de este proyecto, sin embargo, requirió de un estudio mediante simulación de los procesos físicos que participan en la formación de la señal, un análisis de la deposición de energía en el detector y estructura temporal de la señales. A partir de este estudio pude diseñar la electrónica requerida, además de cumplir con los requisitos de consumo de potencia y costo. Para validar la operación del sistema desarrollado hice un experimento en alta montaña, comparando las características de la simulación con las señales reales. Los resultados muestran una buena concordancia entre ambos, además de que me permitieron caracterizar algunas no linealidades del detector.

Finalmente, analicé datos del SciCRT en busca de señales de neutrones solares a partir de \num{2015}, período a partir de cual $1/8$ del detector comenzó a operar de forma regular en el sitio. De este análisis encontré dos incrementos asociados a fulguraciones, ocurridas en Septiembre de \num{2017}. La conclusión del análisis muestra que las partículas registradas pudieron ser originadas durante los eventos solares.

\chapter{Abstract}

Solar neutrons are a unique phenomenon. Relativistic neutrons are produced during intense Solar flares by the interaction between accelerated ions and plasma in the atmosphere of the Sun. Measurement of the their energy spectrum conveys information about the acceleration process, as they travel across the interplanetary medium in their way to Earth unaffected by the magnetic field. There is a strong attenuation of these particles when they interact with the atmosphere and only small number is capable of penetrating deeply. Since neutrons have mass, propagation time depends on energy, therefore distinction between impulsive and continuous emission is only possible measuring their arrival time.

A global network of solar neutron telescopes (SNT) were developed with the sole purpose of studying this phenomenon. The network of telescopes is located on seven mountains near the equatorial line, observing the Sun \SI{24}{\hour} a day, so far being successful detecting several events associated with flares. However, the development of new a type of detector with a larger active volume is essential to improve the statistical significance and energy resolution. This new type of detector is the SciBar cosmic ray telescope  (SciCRT).

In \num{2013} we installed the SciCRT at the top of Sierra Negra (\SI{4600}{\metre}) volcano in Mexico ($\ang{19.0}\mathbf{N}$, $\ang{97.3}\mathbf{W}$), to serve as a new and improved solar neutron and muon telescope. Compared with the previous generation of Solar neutron telescopes, the SciCRT has almost \num{15} times higher target scintillator volume for neutrons, it has also better energy resolution and is sensitive to lower energy neutrons. Furthermore, because the SciCRT records the energy deposition along the path, new schemes of particle identification may be accomplished which in turn improves the sensitivity for solar particles.

Considering that only $3/8$ of the required electronics for the detector is available at the present time, the development of new fast electronics is a priority of our experiment to fully realize the capabilities of SciCRT as an improved SNT. Motivated by this I began designing a new front end electronics for the telescope. To achieve my goal I developed a simulation technique that includes all the relevant process in the signal generation using cosmic ray muons as input to study the energy deposition and timing information. From this I was able to design and build front end electronics units, constraining the problem to find an efficient low power/cost solution. A extensive test of the electronics was done with an experiment at high altitude to extract the features of the real signals and compare them with the simulation. All the results of the experiment are in good agreement with the simulation. Furthermore, from the application of the analysis I was able to determine non-linearities in the response of the detector.

Because $1/8$ of the detector has been working stable since \num{2015}, I analyzed the collected data to search for solar neutron signals associated with large solar flares. From this I identified significant and positive excesses in the event rate of the detector on two different dates on September \num{2017}. After applying a selection criteria (direction, energy and type of particle) I concluded that these increments may be produced by neutrons and $\gamma$-rays coming from the direction of the Sun. I also complemented this studies with a detector performance study.
