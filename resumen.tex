\chapter{Resumen}

Relativistic neutrons are produced during intense Solar flares, traveling across the interplanetary medium in their way to Earth. There is a strong attenuation of these particles when they interact with the atmosphere and only small number is capable of penetrating deeply. Solar neutron telescopes (SNT) were designed and installed on high mountains to observe these events, therefore studying acceleration mechanisms at the solar surface.

The SciBar cosmic ray telescope (SciCRT) is a brand new telescope installed on the top of the Sierra Negra volcano in eastern Mexico, composed of roughly \num{15000} scintillator bars, capable of detecting solar particles with both high efficiency and energy resolution.


In September \num{2017} a series of intense solar flares were registered by the GOES satellite. In this period, the SciCRT registered two signals that could be related with a solar neutron events. In this paper we examine the possible events and use the unique capabilities of the telescope to determine an energy
spectrum of the particles.

n seven mountains near the equatorial line, observing the Sun \SI{24}{\hour} a day, so far being successful detecting several events associated with flares. However, the development of new a type of detector with a larger active volume is essential to improve the statistical significance and energy resolution. This new type of detector is the SciBar cosmic ray telescope  (SciCRT).

The SciBar detector is an active tracker first built for the K2K (KEK-to-Kamioka) long-baseline neutrino experiment and latter used in the SciBooNE experiment at Fermi National Accelerator Laboratory (FNAL). In \num{2013} we installed the SciBar at the top of Sierra Negra (\SI{4600}{\metre}) volcano in Mexico ($\ang{19.0}\mathbf{N}$, $\ang{97.3}\mathbf{W}$) to serve as  the SciCRT, a new and improved solar neutron and muon telescope. Operation at high altitude mountain is a requirement for solar neutron observation due to the attenuation of neutrons in the atmosphere. Compared with the previous generation of Solar neutron telescopes (SNTs), the SciCRT has almost  \num{15} times higher target scintillator volume for neutrons, it has also better energy resolution and is sensitive to lower energy neutrons. Furthermore, because the SciCRT records the energy deposition along the path, new schemes of particle identification may be accomplished which in turn improves the sensitivity for solar particles.

Considering that only $3/8$ of the required electronics for the detector are available at the present time, the development of new fast electronics is a priority of our experiment to fully realize the capabilities of SciCRT as an improved SNT. Motivated by this I began designing of new front electronics for the telescope. To achieve my goal I developed a simulation technique that includes all the relevant process in the signal generation using cosmic ray muons as input to study the energy deposition and timing information. From this I was able to design and build front end electronics units, constraining the problem to find an efficient low power/cost solution. A extensive test of the electronics was done with an experiment at high altitude to extract the features of the real signals and compare them with the simulation. All the results of the experiment are in good agreement with the simulation. Furthermore, from the application of the analysis I was able to determine non-linearities in the response of the detector.

Because $1/8$ of the detector has been working stable since \num{2015}, I analyzed the collected data to search for solar neutron signals associated with large solar flares. From this I identified significant and positive excesses in the event rate of the detector on three different dates on September \num{2017}. After applying a selection criteria (direction, energy and type of particle) I concluded that these increments were produced by neutrons and $\gamma$-rays coming from the direction of the Sun. I also complemented this studies with Monte Carlo simulations on CORSIKA and Geant4. 
