\chapter{Resumen}

Relativistic neutrons are produced during intense Solar flares, traveling across the interplanetary medium in their way to Earth. There is a strong attenuation of these particles when they interact with the atmosphere and only small number is capable of penetrating deeply. Solar neutron telescopes (SNT) were designed and installed on high mountains to observe these events, therefore studying acceleration mechanisms at the solar surface.

The SciBar cosmic ray telescope (SciCRT) is a brand new telescope installed on the top of the Sierra Negra volcano in eastern Mexico, composed of roughly \num{15000} scintillator bars, capable of detecting solar particles with both high efficiency and energy resolution. 


In September \num{2017} a series of intense solar flares were registered by the GOES satellite. In this period, the SciCRT registered two signals that could be related with a solar neutron events. In this paper we examine the possible events and use the unique capabilities of the telescope to determine an energy
spectrum of the particles.
