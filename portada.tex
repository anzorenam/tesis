\begin{titlepage}

\begin{center}

	\vspace{-1cm}
  \includegraphics[height=4.5cm]{unam.pdf}\\[20pt]
	\textbf{\Large UNIVERSIDAD NACIONAL AUTÓNOMA DE MÉXICO}\\
	\large Posgrado en Ciencias de la Tierra\\
	\large Instituto de Geofísica\\
	\large Departamento de Ciencias Espaciales\\[20pt]
	\textbf{\Large Desarrollo de un instrumento para la detección de neutrones solares en la cima
	del Volcán Sierra Negra}\\[20pt]
	\textbf{\Large TESIS}\\
  \large QUE PARA OPTAR POR EL GRADO DE:\\
  \large DOCTOR EN CIENCIAS DE LA TIERRA\\[20pt]
  \large PRESENTA:\\
	\large  Marcos Alfonso Anzorena Méndez\\[20pt]
  \large TUTOR PRINCIPAL:\\
  \large  Dr. José Francisco Valdés-Galicia\\
  \large  Instituto de Geofísica, UNAM\\[20pt]
  \large COMITÉ TUTOR:\\
  \large  Dr. Gustavo Adolfo Medina-Tanco\\
  \large  Instituto de Ciencias Nucleares, UNAM\\
  \large  Dr. José Alberto Ramírez Aguilar\\
  \large  Unidad de Alta Tecnología, UNAM\\[20pt]

  México, D.F. Agosto 2021

\end{center}

%\newpage

%\thispagestyle{empty}
%\setlength{\parskip}{5pt}

%	\Large JURADO ASIGNADO:

%	\vspace*{35pt}
	%\large Presidente: Dr. Francisco García Ugalde

	%\large Secretario: Dra. Lucia Medina Gómez

	%\large Vocal: Dr. José Francisco Valdés-Galicia

	%\large Primer suplente: Dr. Pablo Roberto Perez Alcázar

	%\large Segundo suplente: Dr. Luis Xavier González Méndez

%	\vspace*{35pt}
	%\large Lugar donde se realizó la tesis:

%	\vspace*{15pt}
	%\textbf{\large Telescopio de neutrones solares}

	%\textbf{\large en Sierra Negra, Puebla}

	%\textbf{\large UNAM}
	%\vspace*{35pt}

%\begin{center}

%	\Large TUTOR DE TESIS:

%	\vspace*{35pt}
%	\large \textit{Dr. José Francisco Valdés-Galicia}

%	\large \textit{Investigador Titular C, T.C.}

%\end{center}

\end{titlepage}
