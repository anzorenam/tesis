% !TEX root = tesis.tex

\chapter{Conclusión y trabajo a futuro}
\chaptermark{Conclusión}

La operación del SciCRT será fundamental para comprender los mecanismos de aceleración en el Sol. El nuevo detector, además de su volumen activo mayor, tiene una mayores capacidades para identificar partículas y mejor resolución en energía; lo cual lo hace un excelente instrumento para estimar el espectro de energía de los neutrones solares

The operation of the SciCRT is fundamental to understand the particle acceleration mechanisms at the Sun. Besides the higher effective volume, the new detector enables the particle identification and better energy resolution which in turn will make possible to better estimate the energy spectrum of solar neutrons. Since \num{2015} the telescope has been working stable, with the installation of fast readout DAQ system. Despite this, to fully realize the capabilities of the detector we started designing new FEU.

We developed	a simulation technique that includes all the relevant process in the signal generation using cosmic ray muons as input to study the energy deposition and timing information. From this we were able to design and build FEUs, constraining the problem to find an efficient low power/cost solution. The validation of the methodology was done building a \num{16} channel version of the preamplier/shaper including an FPGA (for discrimination, TDC and data transfer) and testing with a green LED.

A more extensive test was done with an experiment at high altitude and extracting the features of the real signal and comparing them with our simulation. All the results of the experiment are in good agreement with the simulation. Furthermore, from the application of our analysis we were able to determine non-linearities in the response of the detector, an important feature that will be useful in future studies as well as the relation between the TOT signal and the distance to the MAPMT. Furthermore, the study of the attenuation length in the fiber using real data, allows the calibration of the simulation to include optical properties not fully studied in our telescope before. Regarding the timing characteristics of the signal, we found that it is needed to characterize the temperature dependence of the propagation delay in the signal chain as this could lessen the resolution of the detector.

As a final observation, the current study does not include the properties of neutron produced signals in the telescope and it will be of great importance extending this work to cover this subject. In the view that the neutron dedicated blocks in the SciCRT use a low operational gain, the charge deposition could be proportional to the one from muons, however the timing information should change. A complete study of these parameters and corresponding experiment to validate will be the subject of future research.
