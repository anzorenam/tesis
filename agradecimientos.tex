% !TEX root = tesis.tex

\newpage
\thispagestyle{empty}
\vspace*{\fill}

\newpage
\thispagestyle{empty}
\vspace*{\fill}

\begin{verse}

\noindent\emph{Unless I am convinced by the testimony of the Scriptures or by clear reason, I am bound by the Scriptures I have quoted and my conscience is captive to the Word of God. I cannot and I will not recant anything, since it is neither safe nor right to go against conscience. May God help me. Amen.}

\end{verse}

\begin{flushright}
\emph{Martin Lutero ante la Dieta Imperial de Worms}\\
\emph{25 de Mayo de 1521}
\end{flushright}
\vspace*{\fill}

\newpage
\thispagestyle{empty}
\vspace*{\fill}
\begin{center}
\textbf{\Large{Agradecimientos}}
\end{center}

A mi querida UNAM, por una educación pública, laica y gratuita.

Al Dr.\, José Francisco Valdés-Galicia, mi tutor y mentor durante estos doce años de formación. Por abrirme las puertas al mundo científico y enseñarme a andar en él. Agradezco todo su apoyo, dirección e ideas; pero sobretodo agradezco su amistad.

A los miembros de mi comité tutor: Dr.\, Gustavo Adolfo Medina-Tanco y Dr.\, José Alberto Ramírez Aguilar. Por todos los comentarios y observaciones para enriquecer mi investigación. En especial quiero agradecer a Dr.\, Medina-Tanco y al M.\,I.\, Juan Carlos Sánchez Balanzar por ayudarme a realizar el experimento en las instalaciones del Instituto de Ciencias de Nucleares.

A cada uno de los miembros del jurado, Dr.\, Luis Xavier González Méndez, Dr.\, Luis Manuel Villaseñor Cendejas, Dra.\, Karen Salome Caballero Mora y Dr.\, Ernesto Ortiz Fragoso, por su valiosa revisión y comentarios que me ayudaron a mejorar esta investigación.

Agradezco a Dra.\, Karen Caballero y el equipo del LARCAD por permitirme utilizar el cluster para correr las simulaciones.

De manera especial quiero agradecer a Dr.\, Ernesto Ortiz por toda su asesoría a través de estos años, su colaboración en mi formación científica y amistad.

También quiero agradecer el apoyo de Dr.\, Yutaka Matsubara.

A mis demás compañeros/amigos del grupo de Rayos Cósmicos: Ing.\, Octavio Musalem, Fís.\, Alejandro Hurtado, M.\,I.\, Rocío, Taylor, por la ayuda que me proporcionaron en el desarrollo de esta tesis. En particular agradezco a Rocío por enseñarme a usar Geant4. A Octavio, Alejandro y Roberto les agradezco todo su apoyo durante mi estancia en la estación de RC y el trabajo que de forma directa o indirecta hace posible nuestra investigación. Además, agradezco a Alejandro y Roberto por su colaboración durante los experimentos en Sierra Negra y su ayuda para construir los prototipos.

Finalmente quiero agradecer al Instituto Nacional de Astrofísica Óptica y Electrónica (INAOE) por permitirnos instalar el SciCRT en Sierra Negra y darnos todas las facilidades para el experimento.

\newpage
\thispagestyle{empty}
\vspace*{\fill}
\begin{center}
\textbf{\Large{Dedicatoria}}
\end{center}

Escribir esta tesis doctoral, la cual marca el inicio de una nueva etapa de mi vida, es una de las experiencias más agradables que he tenido. No tanto por los frutos obtenidos a partir de la misma, sino por permitirme ser testigo del amor y apoyo de las personas que están a mi lado. Debido a esto, la conclusión más concisa de este trabajo es: \emph{la cura de la soberbia es la necesidad}.

Quiero agradecer a mi amada esposa (y colega) Rocío por compartir conmigo la experiencia de la vida. Por compartir tus sueños y ayudarme alcanzar los míos. Porque juntos atravesamos los momentos más amargos, y no me soltaste de la mano. Agradezco todas las palabras de ánimo, risas y largas discusiones; cada taza de café\ldots\, las horas de desvelo. Gracias porque con tu amor me haz hecho una mejor persona.

Agradezco a mis padres todo el amor y fortaleza que me han dado: sus cuidados y consejos me han hecho el hombre que soy. Me dieron un hogar, una familia y compartieron conmigo el \emph{más dulce tesoro}. Tengan la seguridad de que nunca voy a olvidar donde está mi casa. A mi padre agradezco sus enseñanzas y a mi madre su ternura.

Para Enrique, Miguel, Elisa, Ariel y Zetzuko; mi más profundo aprecio. Hemos caminado juntos, nos hemos soportado, nos hemos caído: sigamos esforzándonos hasta la meta. Los amo mucho y son parte de mi. A mi hermano Enrique, por ser un gran ejemplo para mi. A los más jóvenes: Kassandra, Isai, Shelomi y Eliane les agradezco toda su alegría y todo su cariño para mi.

A mi segunda Familia, la que me adoptó. A Heriberta y Francisco les estoy infinitamente agradecido. Me abrieron su casa, se mostraron como padres para mi; aunque yo era un desconocido. A Lorena, Patricia, Georgina, Alberto, Paola, Isaac y Abraham. Gracias por todas las risas, largas pláticas, aventuras y proyectos\ldots\, En cada uno de ustedes encontré una amiga/amigo especial.

\vspace*{\fill}
